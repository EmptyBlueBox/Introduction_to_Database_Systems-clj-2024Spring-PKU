\documentclass[12pt]{article} % 12pt 为字号大小 UTF8
\usepackage{amssymb,amsfonts,amsmath,amsthm}
%\usepackage{fontspec,xltxtra,xunicode}
%\usepackage{times}

%----------
% 定义中文环境
%----------

\usepackage{xeCJK}

% \setCJKmainfont[BoldFont={SimHei},ItalicFont={KaiTi}]{SimSun}
% \setCJKsansfont{SimHei}
% \setCJKfamilyfont{zhsong}{SimSun}
% \setCJKfamilyfont{zhhei}{SimHei}

% \newcommand*{\songti}{\CJKfamily{zhsong}} % 宋体
% \newcommand*{\heiti}{\CJKfamily{zhhei}}   % 黑体


%----------
% 版面设置
%----------
% %首段缩进
% \usepackage{indentfirst}
% \setlength{\parindent}{2.1em}
% 设置段落不缩进
\setlength{\parindent}{0pt}

%行距
\renewcommand{\baselinestretch}{1.2} % 1.4倍行距

%页边距
\usepackage[a4paper]{geometry}
\geometry{verbose,
  tmargin=3cm,% 上边距
  bmargin=3cm,% 下边距
  lmargin=3cm,% 左边距
  rmargin=3cm % 右边距
}


%----------
% 其他宏包
%----------
%图形相关
\usepackage[x11names]{xcolor} % must before tikz, x11names defines RoyalBlue3
\usepackage{graphicx}
\usepackage{pstricks,pst-plot,pst-eps}
\usepackage{subfig}
\def\pgfsysdriver{pgfsys-dvipdfmx.def} % put before tikz
\usepackage{tikz}

%原文照排
\usepackage{verbatim}
\usepackage{float}

%网址
\usepackage{url}

%文本格式
\usepackage{ulem} % 用法:\uline{}下划线,\uwave{}波浪线,\sout{}删除线
\usepackage{pifont} % 用法:\ding{数字}代表数字被圈起来
\usepackage{enumitem} % 使用enumitem宏包自定义列表样式
\usepackage{hyperref} % 不仅可以帮助你插入链接,还能让这些链接在生成的PDF文档中是可点击的,提高文档的互动性,用法:\url{http://www.example.com}

% 代码块
\usepackage{verbatim}

%----------
% 习题与解答环境
%----------
\usepackage{mdframed}
\newenvironment{problem}[2][Problem]
    { \begin{mdframed}[backgroundcolor=gray!20] \textbf{#1 #2} \\}
    {  \end{mdframed}}

\newenvironment{solution}
    { \begin{mdframed} \textbf{Solution:} \\}
    {  \end{mdframed}}

    % 我的自定义
%----------

\newcommand{\horrule}[1]{\rule[0.5ex]{\linewidth}{#1}} 	% Horizontal rule

\renewcommand{\refname}{参考文献}
\renewcommand{\abstractname}{\large \bf 摘\quad 要}
\renewcommand{\contentsname}{目录}
\renewcommand{\tablename}{表}
\renewcommand{\figurename}{图}

\setlength{\parskip}{0.4ex} % 段落间距

\usepackage{enumitem}
\setenumerate[1]{itemsep=0pt,partopsep=0pt,parsep=\parskip,topsep=5pt}
\setitemize[1]{itemsep=0.4ex,partopsep=0.4ex,parsep=\parskip,topsep=0.4ex}
\setdescription{itemsep=0pt,partopsep=0pt,parsep=\parskip,topsep=5pt}


%==========
% 正文部分
%==========

\begin{document}

\title{
{\normalfont\normalsize\textsc{
Peking University\\
Introduction to Database Systems, Spring 2024 \\[25pt]}}
\horrule{0.5pt}\\
\sffamily{第七章\ 关系规范化\\课后作业}
\horrule{1.8pt}\\[20pt]
}
\author{梁昱桐\quad 2100013116\\lyt0112@outlook.com}
% \date{} % 若不需要自动插入日期,则去掉前面的注释;{ } 中也可以自定义日期格式

\begin{titlepage}
\maketitle
\vspace{30pt}

% \begin{abstract}
% \normalsize \ \ 这是中文摘要。大概写满这一页可以了。摘要又称概要、内容提要。摘要是以提供文献内容梗概为目的,不加评论和补充解释,简明、确切地记述文献重要内容的短文。其基本要素包括研究目的、方法、结果和结论。具体地讲就是研究工作的主要对象和范围,采用的手段和方法,得出的结果和重要的结论,有时也包括具有情报价值的其它重要的信息。\\[5pt]
% \indent \ \ \textbf{关键词}:图卷积神经网络,复杂网络,表示学习
% \end{abstract}

\thispagestyle{empty}
\end{titlepage}

% \tableofcontents
% \thispagestyle{empty}

\newpage
\setcounter{page}{1}

\begin{problem}{1}
一个全是主属性的关系模式最高一定可以达到第几范式?
\end{problem}

\begin{solution}
3NF
\end{solution}

\begin{problem}{2}
一个全码的关系模式最高一定可以达到第几范式?
\end{problem}

\begin{solution}
BCNF
\end{solution}

\begin{problem}{3}
任何一个二目关系模式 $R(A,B)$ 一定属于 BCNF 吗
\end{problem}
\begin{solution}
不一定.

考虑一个关系模式$R(A,B)$,其中A是主属性,B是非主属性.
如果存在一个非平凡的函数依赖 $A\rightarrow B$ ,且A不是候选键,那么这个关系模式就不符合BCNF,因为函数依赖$A\rightarrow B$违反了BCNF的定义.
\end{solution}

\begin{problem}{4}
一个只有一个候选码的3NF关系模式是 BCNF 的吗
\end{problem}
\begin{solution}
不一定.

即使一个关系模式只有一个候选码,如果它的某些非平凡函数依赖是由非候选码决定的,那么它就不符合BCNF.
\end{solution}

\begin{problem}{5}
一个候选码全是单属性的关系模式最高一定可以达到第几范式?
\end{problem}
\begin{solution}
2NF
\end{solution}

\begin{problem}{6}
多值依赖和保持无损连接的模式分解之间的关系是什么?
\end{problem}
\begin{solution}
当我们进行模式分解时,需要考虑保持无损连接的原则,确保分解后的模式仍然能够保留原始数据的完整性和完整性,满足多值依赖的要求,以便通过连接操作重建原始关系.
\end{solution}

\begin{problem}{7}
BCNF分解算法是如何保证分解是无损的?
\end{problem}
\begin{solution}
确定函数依赖,确定超键,识别非平凡的函数依赖,选择违反BCNF的函数依赖,应用分解操作并重复,直到所有的关系模式都满足BCNF.
\end{solution}

\begin{problem}{8}
3NF分解算法的第3步,如何确定此时关系模式已经是3NF的了?
\end{problem}
\begin{solution}
检查所有非平凡的传递依赖是否被消除.
\end{solution}

\begin{problem}{题目一}
\[  
\mathrm{R(ABCDE),F=\{AB~\to~C,B~\to~D,CD~\to~E,CE~\to~B,AC~\to~B\}}
\]
\end{problem}
\begin{solution}
1.	给出其候选码 AB AC 
2.	判断范式级别 1NF
3.	分别给出保持无损连接和函数依赖的分解 R1(ACD), R2(BCE), R3(BD)
\end{solution}

\begin{problem}{题目二}
R(ABCDE), 给出下面函数依赖集在S(ABCD)上的投影
\[\mathrm{F=\{AB~\to~D,AC~\to~E,BC~\to~D,D~\to~A,E~\to~B\}}\]
\end{problem}
\begin{solution}
\[F_S=\{AB\to D,BC\to D,D\to A\}\]
\end{solution}

\begin{problem}{题目三}
关系模式R(BCDFGH)
其函数依赖集为
\[\{\mathrm{BG~\to~CD,~G~\to~F,CD~\to~GH,C~\to~FG,F~\to~D}\}\]
给出其同时保持函数依赖和无损的3NF分解
\end{problem}
\begin{solution}
{BGCD},{GF},{CG},{CDH},{DF}
\end{solution}

\begin{problem}{题目四}
  R(ABCD)上成立函数依赖A → BCD和多值依赖B →→ C
    判断R的范式级别  
  (提示:小心掉坑,还有哪些隐藏的函数依赖?)
\end{problem}
\begin{solution}
  3NF
\end{solution}

\begin{problem}{题目五}
  给出判断关系r(ABC)上多值依赖A →→ B是否成立的关系代数和SQL语句
\end{problem}
\begin{solution}
\[\begin{aligned}r_1&=\pi_{A,B}(r)\\r_2&=\pi_{A,C}(r)\\r_3&=r_1\bowtie r_2\end{aligned}\]

\[\text{若 }r_{3}=r\text{,则多值依赖 }A\to\to B\text{ 成立。}\]

\begin{verbatim}
-- Step 1: 投影出 A 和 B 列
CREATE TABLE r1 AS
SELECT DISTINCT A, B
FROM r;

-- Step 2: 投影出 A 和 C 列
CREATE TABLE r2 AS
SELECT DISTINCT A, C
FROM r;

-- Step 3: 自然连接 r1 和 r2
CREATE TABLE r3 AS
SELECT r1.A, r1.B, r2.C
FROM r1
NATURAL JOIN r2;

-- Step 4: 检查 r3 是否等于原表 r
SELECT *
FROM r
EXCEPT
SELECT *
FROM r3;

SELECT *
FROM r3
EXCEPT
SELECT *
FROM r;
\end{verbatim}

\end{solution}

\begin{problem}{题目六}
  给出函数依赖集{ABCD→E,E→D,A→B,AC→D}的最小覆盖
\end{problem}
\begin{solution}
  {AC→E,E→D,A→B}
\end{solution}

% \newpage
% \bibliographystyle{plain}
% \bibliography{ref}


\end{document}
